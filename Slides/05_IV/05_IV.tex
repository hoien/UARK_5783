\documentclass[aspectratio=169,t,11pt,table]{beamer}
\usepackage{../../slides,../../math}
\definecolor{accent}{HTML}{2B5269}
\definecolor{accent2}{HTML}{9D2235}

\title{Topic 5: Instrumental Variables}
\subtitle{\it  ECON 5783 — University of Arkansas}
\date{Fall 2024}
\author{Prof. Kyle Butts}

\begin{document}

% -----------------------------------------------------------------------------
\begin{frame}[noframenumbering,plain]
\maketitle

% \bottomleft{\footnotesize $^*$A bit of extra info here. Add an asterich to title or author}
\end{frame}
% -----------------------------------------------------------------------------

\section{Introducing Instrumental Variables}

\begin{frame}{Instrumental Variables}
  Instrumental Variables are one of the oldest `causal' identification strategies
  \begin{itemize}
    \item Their roots go back to the early 1900s where economics was primarily the study of particular markets
    
    \item Demand and supply curves were a new phenomenon and economists wanted to try and estimate them 
  \end{itemize}
\end{frame}

\begin{frame}{Supply and Demand Curves}
  Consider a simple supply and demand curve setup
  \begin{align}
    \text{quantity}_{d} &= \alpha_{d} + \text{price}\gamma_{d} + u_{d} \\
    \text{quantity}_{s} &= \alpha_{s} + \text{price}\gamma_{s} + u_{s}
  \end{align}
  Here, this is the theoretical relationship between prices and quantity \emph{at the same point in time} (potential outcomes)
  \begin{itemize}
    \item Equilibrium is determined by the $\text{price}$ such that $\text{quantity}_{d} = \text{quantity}_{s}$
  \end{itemize}

  \bigskip
  Note that `market conditions' (e.g. preferences, production technologies, etc.) are implicitly embedded in the coefficients
\end{frame}

\begin{frame}{Observed market outcomes}
  What we observe is a set of markets $t$ with their corresponding price and quantities $(p_t, q_t)$

  \bigskip
  The problem here is that market-shocks could move both the demand curve and the supply curve
  \begin{itemize}
    \item[$\rightarrow$] The data is not tracing `along' the demand curve or the supply curve
  \end{itemize}
\end{frame}

\imageframe{figures/ex_endogeneity_problem.pdf}

\begin{frame}{Tracing out the demand curve}
  Phillip Wright wrote a book in 1928 about `animal and vegetable oils'
  \begin{itemize}
    \item he was trying to argue that recent tariffs had negatively impacted the market
  \end{itemize}

  \bigskip
  In this book, there is a now famous ``Appendix B'' proposing the first IV estimator
  \begin{itemize}
    \item There's debate who wrote this chapter: father or son. 
    
    \item It was likely the son; see Stock and Trebbi, 2003
  \end{itemize} 
\end{frame}

\begin{frame}{Identifying movement along one curve}
  As we saw in the previous graph, we can not take two points $(p_1, q_1)$ and $(p_2, q_2)$ and know if these fall on the same demand curve or the same supply curve

  \bigskip
  Say somehow we \emph{know} that one of the curves did not change from market $1$ to market $2$ (say from one day to the next)
  \begin{itemize}
    \item Then our two points both must fall on the curve that did not shift (equilibrium conditions)
  \end{itemize}
\end{frame}

\imageframe{figures/ex_supply_shifters.pdf}
\imageframe{figures/ex_demand_shifters.pdf}

\begin{frame}{``Demand Shifters'' and ``Supply Shifters''}
  Wright proposed to use a variable that shifts only one of the curves and not the other, what we now call instruments, to estimate the demand/supply curve:

  \begin{itemize}
    \item E.g. a demand shifter being the change in the price of a substitute good
    \item E.g. a supply shifter being a change in rain from one year to the next
  \end{itemize}

  \pause
  \bigskip 
  Since these `shifters' only affect one curve and not the other, then we can leverage these to estimate the other curve (e.g. demand shifter to estimate supply) 
\end{frame}

\begin{frame}{Fulton Fish Market}
  Angrist, Graddy, and Imbens (2000) give a great example of this. 

  \bigskip
  Graddy, then a graduate student, woke up every morning before sunrise and traveled from New Jersey to the Fulton Fish Market in downtown Manhattan
  \begin{itemize}
    \item Recorded data on the price and quantity sold of fish
    
    \item Also recorded each day's ocean-weather, an important supply curve shifter
  \end{itemize}
\end{frame}

\imageframe{figures/fulton_fish_market.jpg}

\begin{frame}{Fulton Fish Market IV Estimator}
  Let $Z_t$ denote the weather observed in market $t$ (e.g. $Z_t$ is the wind speeds on the ocean). 
  
  \bigskip
  The idea of the IV estimator we will present is to do two things:
  \begin{enumerate}
    \item First, see how weather, $Z_t$, impacts the (log) market price, $\log(p_t)$
    
    \item Second, see how weather, $Z_t$, impacts the (log) market quantity, $\log(q_t)$
  \end{enumerate}

  \bigskip
  Since the demand elasticity is $\frac{\partial \log(q_t)}{\partial \log(p_t)}$, we can take the ratio of these two quantities to estimate the demand elasticity
\end{frame}

\begin{frame}{Fulton Fish Market IV Estimator}
  Our IV estimator can be written as
  $$
    \hat{\tau}_{\texttt{IV}} = \frac{\cov{Z_t, q_t}}{\cov{Z_t, p_t}}
  $$

  \begin{itemize}
    \item Weather causes a shift in $q_t$, $\cov{Z_t, q_t}$, and a shift in $p_t$, $\cov{Z_t, p_t}$
  \end{itemize}

  \bigskip
  Our IV estimate compares how much weather changes the quantity sold in equilibrium to how much weather changes the price to estimate the demand curve
\end{frame}

\begin{frame}{Fulton Fish Market IV Estimator}
  \vspace{-\bigskipamount}
  $$
    \hat{\tau}_{\texttt{IV}} = \frac{\cov{Z_t, q_t}}{\cov{Z_t, p_t}}
  $$

  \bigskip
  Ideally these shifts come from only the supply curve moving; i.e. $Z_t$ has no effect on $\text{quantity}_t^d$ except from the price
  \begin{itemize}
    \item E.g. this requires fish demand not change with the weather, otherwise both curves would be moving
  \end{itemize}
\end{frame}

\section{What makes a good instrument?}

\begin{frame}{Generalizing the IV Estimator}
  What are the ``essential'' ideas of this IV estimator? Say you want to know the causal effect of $X_i$ on some variable $y_i$. We are concerned there are some other variables (in the error term, $\varepsilon_i$) that determine both $X_i$ and $y_i$.

  That is:
  $$
    y_i = X_i \beta + \varepsilon_i,
  $$
  where $\expec{\varepsilon_i}{X_i} \neq 0$
\end{frame}

\begin{frame}{Generalizing the IV Estimator}
  \vspace{-\bigskipamount}
  $$
    y_i = X_i \beta + \varepsilon_i \ \text{ where }\ \expec{\varepsilon_i}{X_i} \neq 0
  $$

  \bigskip
  Our IV Estimator compares the change in $y_i$ induced by $Z_i$ to the change in $X_i$ induced by $Z_i$ to estimate the slope parameter:
  $$
    \frac{\ \Delta y \text{\ \ \  \footnotesize induced by Z} }{ \Delta X \text{\ \ \footnotesize induced by Z} }
  $$
\end{frame}

\begin{frame}{IV Requirements}
  \vspace{-\bigskipamount}
  $$
    y_i = X_i \beta + \varepsilon_i \ \text{ where }\ \expec{\varepsilon_i}{X_i} \neq 0
  $$
  
  \bigskip
  We want an instrument $Z_i$ that does two things:
  \begin{enumerate}
    \item {\color{blue} (Relevancy Condition)} The instrument should cause a change in $X_i$, so that we could trace out the subsequence impact on $y_i$ 

    \pause
    \medskip
    \item {\color{raspberry} (Exclusion Restriction)} The instrument $Z_i$ should change $y_i$ \emph{only through} changing $X_i$ 
  \end{enumerate}

  % \pause
  % \bigskip
  % The first tends to be the easier of the two to satisfy: we need a `shifter' of $X_i$. The second is the hard part...
\end{frame}

\begin{frame}{IV Estimand}
  \vspace{-\bigskipamount}
  \begin{align*}
    \hat{\beta}_{\texttt{IV}} &= 
    \frac{\cov{Z_i, y_i}}{\cov{Z_i, X_i}} \\
    \pause
    &= \frac{\cov{Z_i, X_i \beta + \varepsilon_i}}{\cov{Z_i, X_i}} \\
    \pause
    &= \frac{\cov{Z_i, X_i} \beta + \cov{Z_i, \varepsilon_i}}{\cov{Z_i, X_i}} \\
    \pause
    &= \beta + \frac{\cov{Z_i, \varepsilon_i}}{\cov{Z_i, X_i}}
  \end{align*}
\end{frame}

\begin{frame}{IV Requirements}
  \vspace{-\bigskipamount}
  \begin{align*}
    \hat{\beta}_{\texttt{IV}} = \frac{\cov{Z_i, X_i} \beta + \tcbhighmath{ \cov{Z_i, \varepsilon_i} }}{ \tcbhighmath[colback = bgBlue]{\cov{Z_i, X_i}} }
  \end{align*}

  % \devgrid 
  \begin{tikzpicture}[remember picture, overlay]
    \node [anchor = south, text width = 0.23\textwidth] at (page cs:0.56,0.22) {
      \begin{center}
        \setstretch{0.8}
        {\footnotesize\color{raspberry} \emph{Exclusion Restriction}
        
        $= 0$}
      \end{center}
    };
    
    \node [anchor = north, text width = 0.23\textwidth] at (page cs:0.47,0.32) {
      \begin{center}
        \setstretch{0.8}
        {\footnotesize\color{blue} \emph{Relevancy Condition} 
        
        not dividing by $0$}
      \end{center}
    };
  \end{tikzpicture}

  \bigskip\bigskip
  \begin{enumerate}
    \item {\color{blue} Relevancy Condition} requires $Z_i$ to actually shift $X_i$
    
    \item {\color{raspberry} Exclusion Restriction} requires $Z_i$ to be uncorrelated with other drivers of $y_i$
  \end{enumerate}

  \pause
  \bigskip
  The first tends to be the easier of the two to satisfy: we need a `shifter' of $X_i$. The second is the hard part...
\end{frame}

% \begin{frame}{Exclusion Restriction}
%   \begin{center}
%     ``The instrument $Z_i$ should change $y_i$ \emph{only through} changing $X_i$''
%   \end{center}
% 
%   \bigskip
%   $\cov{Z_i, y_i}$ will pick up on two things:
% 
%   \bigskip
%   \begin{enumerate}
%     \item The change on $y_i$ induced by a change in $X_i$: $\cov{Z_i, X_i} \beta$
%     \begin{itemize}
%       \item Dividing this by $\cov{Z_i, X_i}$ will identify the effect
%     \end{itemize}
% 
%     \pause
%     \medskip
%     \item The relationship between the instrument and omitted variables: $\cov{Z_i, \varepsilon_i}$
%     \begin{itemize}
%       \item We want this to be 0
%     \end{itemize}
%   \end{enumerate}
% \end{frame}
% 


\begin{frame}{Rainfall IV Example}
  For example, consider trying to estimate on family income on the years of schooling children receive in the developing world. Papers have used rainfall as an instrument for income
  \begin{itemize}
    \item The idea is that rainfall is `random' year over year and so that creates `good variation' in family income
  \end{itemize}
\end{frame}

\begin{frame}{Rainfall IV Example}
  In essence, the IV estimator will compare families that had a good rainfall year (and hence more income than normal) to families with a bad rainfall year (and hence less money than expected)
  \begin{itemize}
    \item If high-rainfall families go to school at higher rates, the IV estimator attributes this to higher income
  \end{itemize}
\end{frame}

\begin{frame}{Rainfall IV Exclusion Restriction failure}
  The exclusion restriction assumes that rainfall only affects the school attendance rate \emph{only through} increasing family incomes
  \begin{itemize}
    \item Is this plausible?
  \end{itemize}

  \pause
  \bigskip
  Sarsons (2015, JDE) finds that lower rainfall in developing countries increases conflict between villages
  \begin{itemize}
    \item This means we can't say if the higher school attendance in high rain areas is due to more income from better crop-yields \emph{or} due to lower likelihood of conflict
  \end{itemize}
\end{frame}

\section{Two-stage least squares estimator}

\begin{frame}{Canonical IV Setup}
  The canonical IV setup is as follows:
  \begin{align*}
    y_i &= X_i \beta + \varepsilon_i \\
    X_i &= Z_i \pi + u_i,
  \end{align*}
  where $Z_i$ is our instrument and $X_i$ is the variable of interest

\end{frame}

\begin{frame}{Canonical IV setup}
  \vspace{-\bigskipamount}
  $$
    y_i = X_i \beta + {\color{raspberry} \varepsilon_i} \ \text{ and }\  X_i = Z_i {\color{blue} \pi}  + u_i
  $$

  \bigskip
  \only<1>{
    In terms of this model, our problem is that $\expec{X_i \varepsilon_i} \neq 0$, i.e. that there are unobservables that are correlated with $X_i$ and have an impact on $y_i$
  }

  \only<2>{
    We instead use an instrument $Z_i$. Our two requirements for the instrument can be written as follows: 
    \begin{enumerate}
      \item {\color{blue} (Relevance)} $\pi \neq 0$
      \begin{itemize}
        \item This is testable with $t$-test that $\pi = 0$ 
        \item Later, problems with `weak' instruments where $\pi \approx 0$ (relative to noise)
      \end{itemize} 
      
      \medskip
      \item {\color{raspberry} (Exclusion)} $\expec{\varepsilon_i Z_i} = 0$
      \begin{itemize}
        \item Fundamentally untestable (it is an assumption)
      \end{itemize}
    \end{enumerate}
  }
\end{frame}

\begin{frame}{Canonical IV setup}{Exclusion Restriction}
  \vspace{-\bigskipamount}
  $$
    y_i = X_i \beta + \varepsilon_i \ \text{ and }\  X_i = Z_i \pi + u_i
  $$

  \bigskip
  For example, say there is some variable $\mu_i$ that is part of the error term $\varepsilon_i = \mu_i + \upsilon_i$
  \begin{itemize}
    \item If $Z_i$ is correlated with $\mu_i$ (or affects $\mu_i$), then our exclusion restriction fails
  \end{itemize}
\end{frame}

% TODO
% \begin{frame}{Instrument Examples}{Example 1}
%   
% \end{frame}
% 
% \begin{frame}{Instrument Examples}{Example 2}
%   
% \end{frame}
% 
% \begin{frame}{Instrument Examples}{Example 3}
%   
% \end{frame}

\begin{frame}{Two-stage least squares (2SLS)}
  \vspace{-\bigskipamount}
  $$
    y_i = X_i \beta + \varepsilon_i \ \text{ and }\  X_i = Z_i \pi  + u_i
  $$

  The previous estimator is identical to the two-stage least squares estimator:
  $$
    \hat{\beta}_{\texttt{2SLS}} = \frac{ (X_i P_Z)' P_Z y_i }{ (X_i P_Z)' (P_Z X_i) } = \frac{\hat{X}_i' \hat{y}_i}{\hat{X}_i' \hat{X}_i},
  $$
  where $P_Z = Z (Z'Z)^{-1} Z'$ is the projection-matrix from ordinary least squares regression
  
  \pause
  \begin{itemize}
    \item This is numerically identical when $Z$ is a single instrument to our previous estimator $\cov{Z_i, y_i} / \cov{Z_i, X_i}$
  \end{itemize}
\end{frame}

\begin{frame}{Two-stage least squares Estimator (2SLS)}
  \vspace*{-\bigskipamount}
  $$
    \hat{\beta}_{\texttt{IV}} = \frac{\hat{X}_i' \hat{y}_i}{\hat{X}_i' \hat{X}_i},
  $$

  \bigskip
  Can think of this as being done in two-stages (hence the name):
  \begin{itemize}
    \item Predict $y_i$ and $X_i$ using the instrument $Z_i$ via separate regressions 
    \item Regress $\hat{y}_i$ on $\hat{X}_i$
  \end{itemize}

  \pause
  \bigskip
  Mostly Harmless Econometrics: \emph{``Intuitively, conditional on covariates, 2SLS retains only the variation in $X$ that is generated by quasi-experimental variation; that is generated by the instrument $Z$''}
\end{frame}

\begin{frame}{Two-stage least squares Estimator (2SLS)}
  \alert{Reduced-form:} $y_i = Z_i \gamma + u_i$

  \bigskip
  \alert{First-stage:} $X_i = Z_i \pi + v_i$

  \bigskip
  $$
    \beta_{\texttt{2SLS}} \equiv \frac{\hat{\gamma}}{\hat{\pi}}
  $$
\end{frame}

\begin{frame}{OLS with Controls vs. IV}
  The Frisch-Waugh-Lovell (FWL) theorem helped us understand what control variables do in a regression of $y_i$ on a varaible $X_i$ and controls $W_i$
  \begin{itemize}
    \item Use $W_i$ to \emph{predict} $X_i$ and $y_i$ and remove that predicted variation
    
    \item Regress $y_i - \hat{y}_i$ on $X_i - \hat{X}_i$
  \end{itemize}
\end{frame}

\begin{frame}{OLS with Controls vs. IV}
  Both estimators wants to use variation in $X_i$ that is `plausibly exogenous'; sometimes called `quasi-experimental variation'. 
  
  \bigskip
  The IV estimator and OLS with controls try to get at `good' variation in $X_i$ in different ways:

  \bigskip
  \begin{enumerate}
    \item OLS `removes bad variation'
    \begin{itemize}
      \item Use controls that you think pick up on variables (in $\varepsilon_i$) that are correlated with $X_i$
      
      % \begin{itemize}
      %   \item Remove that variation in $X_i$
      % \end{itemize}
    \end{itemize}
    
    \medskip
    \item IV `isolates good variation'
    \begin{itemize}
      \item Use an instrument that you think is \emph{not} correlated with $\varepsilon_i$ but that shift $X_i$ 
      
      % \begin{itemize}
      %   \item Only use that variation in $X_i$
      % \end{itemize}
    \end{itemize}
  \end{enumerate}
\end{frame}



\begin{frame}{Weak IV}
  \vspace{-\bigskipamount}
  $$
    y_i = X_i \beta + \varepsilon_i \ \text{ and }\  X_i = Z_i \pi  + u_i
  $$
  
  \bigskip
  When our estimated $\hat{\pi} \approx 0$ we have a `weak instruments' problem
  \begin{itemize}
    \item I.e. close to zero relative to the noise
  \end{itemize}

  \pause
  \bigskip
  This means there is a small covariance between $X_i$ and $Z_i$:
  \begin{itemize}
    \item Since $\hat{\beta}_{\texttt{IV}}$ divides by $\cov{X_i, Z_i}$, the estimate is very noisy 
    \begin{itemize}
      \item $1/0.0001 = 10000$ vs. $1/0.00005 = 20000$
    \end{itemize}
  \end{itemize}
\end{frame}

\begin{frame}{Weak IV}
  The rule of thumb is to use an instrument when the $F$-stat on the first-stage is $\geq 10$
  \begin{itemize}
    \item First-stage is $X_i = Z_i \pi + u_i$ 
    
    \item In case of single IV $Z_i$, equivalent to $t$-stat on $Z_i$ to be $\geq \sqrt{10} = 3.16$
  \end{itemize}

  \pause
  \bigskip
  Some recent work shows that 10 might be too small of a lower bound and you may prefer something like $F \geq 50$ ($t$-stat $\geq 7.07$), but it is unsettled debate
\end{frame}

\begin{frame}{Many Instruments}
  Another problem comes in when you have a large number of instruments (relative to sample size)

  \bigskip
  Our 2SLS estimator regresses $P_Z y_i$ on $P_Z X_i$
  \begin{itemize}
    \item Increasing the dimension of $Z$ means we are going to necessarily predict $X_i$ and $y_i$ better and better
    
    \item Our $P_Z y_i \to y_i$ and $P_Z X_i \to X_i$
  \end{itemize}

  \bigskip
  \pause
  $\implies$ that with too many instruments we $\hat{\beta}_{\texttt{2SLS}} \approx \hat{\beta}_{\texttt{OLS}}$
\end{frame}

\begin{frame}{Many Instruments Example}
  One common IV strategy is the `judge-leniency design' setting:
  \begin{itemize}
    \item Defendents are randomly assigned to different judges
    \item Judges vary in how lenient they are in sentencing
  \end{itemize}

  
  \bigskip
  $\implies$ people with similar backgrounds will be randomly convicted/not-convicted
  \begin{itemize}
    \item Use as an instrument a set of dummy variables for each assigned judge (judge 1, judge 2, $\dots$)
  \end{itemize}
\end{frame}

\begin{frame}{Many Instruments}
  The best way to fix the issue of many-instruments is using a JIVE estimator
  \begin{itemize}
    \item For observation $i$, predict $X_i$ using a leave-out regression of $X_j$ on $Z_j$ using all observations besides $i$: $\{ 1, \dots, n \} \setminus \{ i \}$
  \end{itemize}

  \pause
  \bigskip
  Call the predicted values of the leave-out regressions as $\hat{X}_{i, \texttt{lo}}$. Then we have 
  $$
    \hat{\beta}_{\texttt{JIVE}} \equiv \frac{\hat{X}_{i, \texttt{lo}}' y_i}{\hat{X}_{i, \texttt{lo}}' \hat{X}_{i, \texttt{lo}}}
  $$

  \begin{itemize}
    \item $\hat{\beta}_{\texttt{JIVE}}$ avoids the problem of over-fitting on the instruments 
    
    \item See Kolesar (2013, Working Paper)
  \end{itemize}
\end{frame}

% \begin{frame}
%   \vspace{-\bigskipamount}
%   $$
%     y_i = X_i \beta + \varepsilon_i \ \text{ and }\  X_i = Z_i \pi  + u_i
%   $$
%   There is a rule that the $F$-stat that $\textrm{H}_0: \pi = 0$ should be $>= 10$; this corresponds to a $t$-stat of $\sqrt{10} = 3.16$. 
%   \begin{itemize}
%     \item The simulations used to show this relies on two-sided tests ($\pi \neq 0$)
%   \end{itemize}
% 
%   \bigskip
%   Angrist and Kolesar (2024, JOE) show if we assume the sign of $\pi \geq 0$ (or $\leq 0$), 
%   \begin{itemize}
%     \item This let's us use a one-sided test $\implies$ our equivalent of a two-sided $t$-stat of $3.16$ (and hence $F$-stat) be smaller 
%   \end{itemize}
% 
%   \bigskip
%   Relies on researcher only `pursuiting' a project if $\hat{\pi}$ is of the correct sign
% \end{frame}

\section{IV with Heterogeneous Effects}
\begin{frame}{Heterogeneous Effects}
  Up until this slide we have (implicitly) assumed that the marginal causal effect of increasing $X_i$ was given by $\beta$ 
  \begin{itemize}
    \item Assumed to be the same across individuals
  \end{itemize}

  \bigskip
  When we allow for heterogeneous effects, what our 2SLS estimator finds is more complicated to understand
  \begin{itemize}
    \item Hopefully, it is some `reasonable' weighted average of heterogeneous effects, $\hat{\beta}_{\texttt{IV}} \to \sum_i w_i \beta_i$
  \end{itemize}
\end{frame}

\begin{frame}{Angrist-Imbens-Rubin Causal Model}
  Let's see if we can make progress when $D_i$ is a binary variable, $Z_i$ is a binary variable, but allow the treatment effect $y_i(1) - y_i(0)$ to be heterogeneous.

  \bigskip 
  In Angrist, Imbens, and Rubin (1996, JASA), they study military service ($D_i$) on future earnings ($y_i$)
  \begin{itemize}
    \item The first problem we have is that military service is not randomly assigned
  \end{itemize} 
  
  \pause
  \bigskip
  They use the Vietnam-Draft lottery as an instrument ($Z_i$)
  \begin{itemize}
    \item $Z_i$ is randomly assigned by birthday$^\dagger$
  \end{itemize}

  \bottomleft{
    \footnotesize $^\dagger$This has been questioned because the bingo balls were not shuffled enough before drawing 
  }
\end{frame}

\begin{frame}{Imperfect Compliance}
  The Vietnam draft lottery is definitely a shifter: those drafted by lottery were more likely to serve
  \begin{itemize}
    \item But we don't have a real RCT. Some with $Z_i = 0$ serve $D_i = 1$ and some with $Z_i = 1$ do not serve $D_i = 0$
  \end{itemize}

  \bigskip
  \pause
  This is what we call ``imperfect compliance'', i.e. treatment is not a perfect function of the instrument
\end{frame}

\begin{frame}{Potential Outcome Model with Imperfect Compliance}
  To accomodate this, Angrist-Imbens-Rubin framework defines $D_i(Z_i)$ to be the potential outcomes of treatment under both states of the instrument. In our example,
  \begin{itemize}
    \item $D_i(0)$ is whether the person would serve in the world where they \emph{are not} drafted
    \item $D_i(1)$ is whether the person would serve in the world where they \emph{are} drafted
  \end{itemize}
\end{frame}

\begin{frame}{Potential Outcome Model with Imperfect Compliance}
  Outcomes are now a potential outcome of both $Z_i$ and $D_i$: $y_i(D_i, Z_i)$
  \begin{itemize}
    \item Outcomes depend on both whether you were assigned $Z_i = 0, 1$ and whether you are under treatment $D_i = 0, 1$
  \end{itemize}

  \pause
  \bigskip
  If we assume the \emph{exclusion restriction} that $Z_i$ \emph{only} impacts $y_i$ via it's effect on $D_i$, then we can return to $y_i(D_i)$
  \pause
  \begin{itemize}
    \item In the case of the Vietnam Draft lottery, this is plausible assuming that the draft only is correlated with outcomes via causing people to serve
  \end{itemize}
\end{frame}

\begin{frame}{Potential Outcome Model with Imperfect Compliance}
  Our potential outcomes are therefore
  $$
    D_i(Z_i) \ \text{ and } y_i(D_i)
  $$
  Our causal model implicitly says:
  \begin{enumerate}
    \item $Z_i$ might impact $D_i$ which impacts $y_i$
    \item $Z_i$ only impacts $y_i$ via $D_i$
  \end{enumerate}
\end{frame}

\begin{frame}{Characterizing People}
  There are four kinds of people in this model:
  \begin{enumerate}
    \item \textbf{Compliers}: people who react to the instrument as expected
    \begin{itemize}
      \item $D_{i}(1) = 1$ and $D_{i}(0) = 0$
    \end{itemize} 

    \item \textbf{Always-takers}: people who always take the treatment regardless of $Z$
    \begin{itemize}
      \item $D_{i}(1) = 1$ and $D_{i}(0) = 1$
    \end{itemize}

    \item \textbf{Never-takers}: people who never take the treatment regardless of $Z$
    \begin{itemize}
      \item $D_{i}(1) = 0$ and $D_{i}(0) = 0$
    \end{itemize}

    \item \textbf{Defiers}: people who react to the instrument in the wrong direction
    \begin{itemize}
      \item $D_{i}(1) = 0$ and $D_{i}(0) = 1$
    \end{itemize}
  \end{enumerate}
\end{frame}

\begin{frame}{Defiers}
  Defiers are often ruled out as implausible:
  \begin{itemize}
    \item ``I would have served but not if I was drafted''
  \end{itemize}

  \bigskip
  But, be careful, they may not be implausible in other settings!
\end{frame}

\begin{frame}{Characterizing People}
  Ruling out defiers, three types remain:
  \begin{enumerate}
    \item \textbf{Compliers}: people who react to the instrument as expected
    \begin{itemize}
      \item $D_{i}(1) = 1$ and $D_{i}(0) = 0$
    \end{itemize} 

    \item \textbf{Always-takers}: people who always take the treatment regardless of $Z$
    \begin{itemize}
      \item $D_{i}(1) = 1$ and $D_{i}(0) = 1$
    \end{itemize}

    \item \textbf{Never-takers}: people who never take the treatment regardless of $Z$
    \begin{itemize}
      \item $D_{i}(1) = 0$ and $D_{i}(0) = 0$
    \end{itemize}
  \end{enumerate}

  \bigskip
  It is impossible to know who is which in the data
\end{frame}

\begin{frame}{First-stage}
  Our first-stage consists of regressing $D_i$ on $Z_i$ and an intercept:
  $$
    D_i = \alpha + Z_i \pi + u_i
  $$

  \bigskip
  From regression mechanics and since $Z_i$ and $D_i$ are indicators, we know
  \begin{itemize}
    \item $\hat{\alpha}$ is the share of people with $D_i = 1$ with $Z_i = 0$
    
    \item $\hat{\pi}$ is the difference in share of people with $D_i = 1$ with $Z_i = 1$
  \end{itemize}
\end{frame}

\begin{frame}{First-stage}
  \vspace*{-\bigskipamount}
  $$
    D_i = \alpha + Z_i \pi + u_i
  $$

  In the $Z_i = 0$ group, people with $D_i = 1$ must be always-takers
  \begin{itemize}
    \item $\hat{\alpha}$ is the share of always-takers in the population
  \end{itemize}
  
  \pause
  \bigskip
  In the $Z_i = 1$ group, people with $D_i = 1$ are always-takers \emph{or} compliers
  \begin{itemize}
    \item $\hat{\alpha} + \hat{\pi}$ is the share of always-takers or compliers in the population
  \end{itemize}

  \bigskip
  \pause
  $\hat{\pi}$ is our estimated share of compliers
  \begin{itemize}
    \item Share of people who are `pushed' into treatment
  \end{itemize}
\end{frame}

% TODO: Define the two stages explicitly
\begin{frame}{Reduced-form}
  The reduced-form is the regression of 
  $$
    y_i = \gamma + Z_i \delta + u_i
  $$

  Since $Z_i$ is an indicator variable
  \begin{itemize}
    \item $\hat{\gamma} = \expechat{y_i}{Z_i = 0}$
    \item $\hat{\delta} = \expechat{y_i}{Z_i = 1} - \expechat{y_i}{Z_i = 0}$ 
  \end{itemize}
\end{frame}

\begin{frame}{Reduced-form}
  \only<1>{
    We can use the law of iterated expectations to write our estimate $\hat{\delta}$ as a weighted-sum of the three groups:

    \begin{align*}
      \hat{\delta} &= 
      \expec{y_i}{Z_i = 1} - \expec{y_i}{Z_i = 0} \\
      &= \prob{\text{Always}_i} \left( \expec{y_i}{Z_i = 1, \text{Always}_i} - \expec{y_i}{Z_i = 0, \text{Always}_i} \right) \\
      &\quad + \prob{\text{Never}_i} \left( \expec{y_i}{Z_i = 1, \text{Never}_i} - \expec{y_i}{Z_i = 0, \text{Never}_i} \right) \\
      &\quad + \prob{\text{Complier}_i} \left( \expec{y_i}{Z_i = 1, \text{Complier}_i} - \expec{y_i}{Z_i = 0, \text{Complier}_i} \right) \\
    \end{align*}
  }
  \only<2>{
    Switching $y_i$ in each to case to the corresponding potential outcome $y_i(d)$:

    \begin{align*}
      \hat{\delta} &= 
      \expec{y_i}{Z_i = 1} - \expec{y_i}{Z_i = 0} \\
      &= \prob{\text{Always}_i} \left( \expec{y_i(1)}{Z_i = 1, \text{Always}_i} - \expec{y_i(1)}{Z_i = 0, \text{Always}_i} \right) \\
      &\quad + \prob{\text{Never}_i} \left( \expec{y_i(0)}{Z_i = 1, \text{Never}_i} - \expec{y_i(0)}{Z_i = 0, \text{Never}_i} \right) \\
      &\quad + \prob{\text{Complier}_i} \left( \expec{y_i(1)}{Z_i = 1, \text{Complier}_i} - \expec{y_i(0)}{Z_i = 0, \text{Complier}_i} \right) \\
    \end{align*}
  }
  \only<3>{
    From random assignment of $Z_i$, we have:

    \begin{align*}
      \hat{\delta} &= 
      \expec{y_i}{Z_i = 1} - \expec{y_i}{Z_i = 0} \\
      &= \prob{\text{Always}_i} \left( \expec{y_i(1)}{\text{Always}_i} - \expec{y_i(1)}{\text{Always}_i} \right) \\
      &\quad + \prob{\text{Never}_i} \left( \expec{y_i(0)}{\text{Never}_i} - \expec{y_i(0)}{\text{Never}_i} \right) \\
      &\quad + \prob{\text{Complier}_i} \left( \expec{y_i(1)}{\text{Complier}_i} - \expec{y_i(0)}{\text{Complier}_i} \right) \\
    \end{align*}
  }
  \only<4>{
    The always-taker and never-taker terms equal 0, so:

    \begin{align*}
      \hat{\delta} &= 
      \expec{y_i}{Z_i = 1} - \expec{y_i}{Z_i = 0} \\
      &= \prob{\text{Complier}_i} \left( \expec{y_i(1)}{\text{Complier}_i} - \expec{y_i(0)}{\text{Complier}_i} \right) \\
    \end{align*}
  }
\end{frame}

\begin{frame}{Reduced-form}
  \vspace*{-\bigskipamount}
  \begin{align*}
    \hat{\delta} = \expechat{y_i}{Z_i = 1} - \expechat{y_i}{Z_i = 0}
  \end{align*}
  \vspace*{-\bigskipamount}
  \begin{itemize}
    \item For always-takers, $y_i = y_i(1)$ when $Z_i = 1$ and $Z_i = 0$
    \item For never-takers, $y_i = y_i(0)$ when $Z_i = 1$ and $Z_i = 0$
  \end{itemize}

  \bigskip
  Therefore, for all but compliers, the difference equals 0. So, 
  \begin{align*}
    \hat{\delta} = \prob{\text{Complier}_i} \expec{y_i(1) - y_i(0)}{\text{Complier}_i}
  \end{align*}
\end{frame}

\begin{frame}{IV Estimand}
  Our IV Estimand therefore is the ratio 
  \begin{align*}
    \hat{\beta}_{\texttt{2SLS}} &= 
    \frac{\hat{\delta}}{\hat{\pi}} \\
    \pause
    &= \frac{\prob{\text{Complier}_i} \expec{y_i(1) - y_i(0)}{\text{Complier}_i}}{\prob{\text{Complier}_i}} \\
    &= \expec{y_i(1) - y_i(0)}{\text{Complier}_i}
  \end{align*}
 
  \bigskip
  We estimate the average treatment effect among the compliers
  \begin{itemize}
    \item E.g. the average treatment effect among those induced to serve in the military by the lottery
  \end{itemize}
\end{frame}

\begin{frame}{The `local' average treatment effect}
  \vspace*{-\bigskipamount}
  $$
    \hat{\beta}_{\texttt{2SLS}} = \expec{y_i(1) - y_i(0)}{\text{Complier}_i}
  $$

  Angrist, Imbens, and Rubin called this the \alert{Local Average Treatment Effect} 
  \begin{itemize}
    \item In applications where you have some `shifter' of treatment, but imperfect compliance, you want to always think about \emph{who your compliers are}
  \end{itemize}
\end{frame}

\begin{frame}{Example LATEs}
  Angrist and Evans (1998, AER) study the effect of family size on the parent's labor supply
  \begin{itemize}
    \item They use as an instrument for family size whether or not the first two kids are of the same gender (both male / both female)
    
    \item The idea being the baby's genders are random and induces people to try for a third kid
  \end{itemize}

  \bigskip
  Who are the compliers here?
  \pause
  \begin{itemize}
    \item People who want baby's of both genders and do not mind having three kids
  \end{itemize}
\end{frame}

\begin{frame}{Example LATEs}
  Frimmel, Halla, and Winter-Ebmer (2024, JPubE) study the impact of parental divorce on chidlrens' outcomes
  \begin{itemize}
    \item Their instrument is the number of `women in [the father's] relevant age-occupation-group at work are more likely to divorce'
  \end{itemize}

  \bigskip
  Who are the compliers here?
  \pause
  \begin{itemize}
    \item Men who would not have been unfaithful absent the opportunity of meeting more women at work

    \item This example is from Anna Stansbury on \href{https://x.com/annastansbury/status/1841189989163372870}{twitter}
  \end{itemize}
\end{frame}



\begin{frame}{Better LATE than Nothing}

  The fact that in a world with imperfect compliance, IV estimates a Local ATE creates a problem
  \begin{itemize}
    \item How do you compare estimates across studies?
  \end{itemize}
  
  \bigskip
  In general this is hard to do since the compliers in different papers might look different
\end{frame}

\begin{frame}{Comparing LATEs example}
  Consider two different papers estimating returns to education:
  \begin{enumerate}
    \item Card (1995) uses distance to local college as an instrument (find 13\% increase)
    \begin{itemize}
      \item Compliers are those that only attend college if they can be close to home
    \end{itemize}
    
    \bigskip
    \item Zimmerman (2014, JOLE) uses a GPA threshold for for admissions at a large public university (find 22\% increase)
    \begin{itemize}
      \item Compliers are those near the GPA threshhold who would otherwise not go to college
    \end{itemize}
  \end{enumerate}

  \bigskip
  Do the differences in estimates indicate one of the papers is biased? Or are the complier groups different?
\end{frame}

\begin{frame}{Characterizing Compliers}
  Since we can only estimate a LATE, we might not be able to apply our estimates to new settings
  \begin{itemize}
    \item This is called a question of ``external validity'', i.e. whether our estimates could reasonably be applied to other settings
  \end{itemize}
  
  \bigskip
  To see which settings our estimates might reasonably apply to, we want to describe who the compliers are
  \begin{itemize}
    \item Of course, we can not identify compliers since we only see $D_i(1)$ or $D_i(0)$, not both
  \end{itemize}
\end{frame}

\begin{frame}{Characterizing Compliers}
  It turns out we can still characterize compliers by their potential outcomes ($Y_i(0)$ and $Y_i(1)$) and other observables $X_i$
  \begin{itemize}
    \item Comparing $\expec{ \cdot }{\text{Complier}_i}$ to $\expec{ \cdot }$ can maybe shed light on how LATE compares to overall ATE
  \end{itemize}

  \bigskip
  We can do so using a special trick of multiplying a variable with $D_i$ and using that as our outcome variable in 2SLS 
  \begin{itemize}
    \item $Z_i D_i$ equals 0 for untreated units ($D_i = 0$) and equals $Z_i$ for treated units
  \end{itemize}
\end{frame}

\begin{frame}{Characterizing Compliers}
  Three results:
  \begin{enumerate}
    \item 2SLS of $X_i D_i$ on $D_i$ with $Z_i$ as an instrument identifies
    $$
      \expec{X_i}{\text{Complier}_i}
    $$
    
    \item 2SLS of $Y_i D_i$ on $D_i$ with $Z_i$ as an instrument identifies
    $$
      \expec{Y_i(1)}{\text{Complier}_i}
    $$

    \item 2SLS of $Y_i (1 - D_i)$ on $(1 - D_i)$ with $Z_i$ as an instrument identifies
    $$
      \expec{Y_i(0)}{\text{Complier}_i}
    $$
  \end{enumerate} 
\end{frame}

\section{Extensions}

\begin{frame}{Multi-valued instruments}
  We have discussed the identification of LATEs when we have one single binary instrument $Z_i$
  
  \bigskip
  We needed to rule out `defiers', i.e. people who had $D_i(1) = 0$ but $D_i(0) = 1$. 
  \begin{itemize}
    \item This is sometimes called \alert{monotonicity}, i.e. the instrument only moves treatment in one direction (compliers)
    
    \item Angrist and Krueger show that monotonicity and instrument exclusion imply that we identify a LATE
  \end{itemize}
\end{frame}

\begin{frame}{Multi-valued Instruments}
  What about when $Z_i \in \{1, \dots, J\}$ is a multi-valued treatment?
  \begin{itemize}
    \item E.g. judge leniency designs (random assignment of judges)
  \end{itemize}

  \bigskip
  What does the 2SLS estimate of $Y_i$ on $D_i$ using the set of judge indicator variables, $\one{Z_i = j}$, as instruments identify?
  \begin{itemize}
    \item Under \alert{monotonicity}, the 2SLS estimate will a weighted average of each $Z_{i, k}$'s LATE
  \end{itemize}
\end{frame}

\begin{frame}{Judge IV and LATE}
  In the judge IV case, think of $D_i$ as being a potential outcome for each judge
  \begin{itemize}
    \item Monotonicity requires that random assignment of judges can only move $D_i$ in one direction
  \end{itemize}

  \bigskip
  If Judge $A$ is more `strict' than judge $B$, monotonicity requires:
  \begin{itemize}
    \item Everyone judge $B$ would sentence, judge $A$ would also sentence them (and more people)
  \end{itemize}

  \pause
  \bigskip
  \alert{Montonicity:} For all judges, $j$ and $j'$, either $D_i(j) \geq D_i(j')$ or $D_i(j) \leq D_i(j)$ for all $i$
\end{frame}

\begin{frame}{Judge IV and LATE}
  If judge $A$ and judge $B$ disagree on who they think is risky, then monotonicity would fail
  \begin{itemize}
    \item $D_i(A) > D_i(B)$ for some defendents, but $D_i(B) > D_i(A)$ for other defendents
  \end{itemize}

  \bigskip If monotonicity fails, we might not have a weighted avearge of LATEs
\end{frame}


\begin{frame}{Intent-to-treat Effect}
  Say you do not observe the treatment dummy $D_i$, but only the shifter $Z_i$. 
  You can still estimate the reduced-form $\hat{\delta}$, but not the first-stage $\hat{\pi}$.

  \pause
  \bigskip
  The reduced-form estimates 
  $$
    \hat{\delta} = \prob{\text{Complier}_i} \expec{y_i(1) - y_i(0)}{\text{Complier}_i}
  $$

  \bigskip
  You have the LATE times a proportion between 0 and 1. This is called the \alert{Intent-to-treat effect}
\end{frame}

\begin{frame}{Intent-to-treat Effect}
  \vspace*{-\bigskipamount}
  $$
    \hat{\delta} = \prob{\text{Complier}_i} \expec{y_i(1) - y_i(0)}{\text{Complier}_i}
  $$
  
  \bigskip
  If you put bounds on $\prob{\text{Complier}_i}$ (e.g. 10-20\% of people would be compliers), you can give a range of the LATE (between $10 * \hat{\delta}$ and $20 * \hat{\delta}$)
  \begin{itemize}
    \item This is \emph{very ad-hoc}, but something you could do to put ITT into context
    \item Be careful with low compliance (i.e. weak instrument) your estimate can be very sensitive
  \end{itemize}
\end{frame}


\end{document}
    
