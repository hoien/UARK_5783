\documentclass[12pt]{article}
\usepackage{../../lecture_notes}
\usepackage{../../math}
\usepackage{../../uark_colors}

\hypersetup{
  colorlinks = true,
  allcolors = ozark_mountains,
  breaklinks = true,
  bookmarksopen = true
}

\begin{document}
\begin{center}
  {\Huge\bf Final Project}
  
  \smallskip
  {\large\it  ECON 5783 — University of Arkansas}

  \medskip
  {\large Prof. Kyle Butts}
\end{center}

The goal of this assignment is for you to write a ``short paper'' answering an empirical question using microeconometric data (think 4-10 pages). 
A complete project must include:
\begin{itemize}
  \item a few paragraph introduction motivating your question; 
  \item a research question in the form of "the impact of X on Y" (this should lead from your motivation); 
  \item a description of your data and the context; 
  \item your empirical strategy (including why the chosen strategy is valid for the question/data);
  \item present 1 or 2 results;
  \item and a brief discussion on limitations
\end{itemize}

The empirical strategy and the discussion of the limitations are the most important part of this project. 
One takeaway from this class is that estimating causal effects in the real world is hard! 
Therefore, I want you to think carefully about what you end up estimating. 
Does it need to be a perfect, clean-cut causal estimate, no (but also try!). 
But, I want you (1) to write clearly about what you need to assume to be true if you were to call your estimate ``causal'', and (2) to write clearly about whether or not you believe that. 
You will not lose credit if it's not perfectly defensible, so I do not want you to ``lie'' and say your assumption is perfectly defensible if it is not; 

A good place to look at for examples of this would be to scroll publications in \href{https://www.aeaweb.org/journals/aeri}{AER Insights} (highly ranked journal), \href{https://www.sciencedirect.com/journal/economics-letters}{Economic Letters}, or \href{https://www.tandfonline.com/journals/rael20}{Applied Economics Letters}. 
The first journal is very highly ranked (so results might be more complicated and better causally identified) while the other two are solid, but not great, journals. 

As you look at papers, try to do two things:
\begin{enumerate}
  \item Identify topics you find interesting / exciting. This will help you look for data and questions to answer
  
  \item Try to notice how they discuss their empirical strategy and how they discuss the plausibility of their assumptions. 
  Note how they might use figures/tables/prose to defend their arguments.
\end{enumerate}

\bigskip
\subsection*{Coming up with a research idea}

If you would like, you can come to my office hours (listed on the syllabus) to discuss your research idea. 
We can talk through empirical strategy and how to probe if the necessary assumptions are plausible. 
If you are having a hard time, we can also try to look for datasets to use in the project.

One good way to find interesting policies to study is to open up magazines / newspaper articles from the 2010s where data is available and easy to access and enough time has passed to actually let the policy have impacts!
Reading articles about newly passed policies probably will not work because the policies (probably) will not have been rolled out yet.


\bigskip
\subsection*{Differences in Assignment for PhD vs. MS students}

\begin{itemize}
  \item If you are a Master's level student, you may work on this assignment solo or in a group of 2. You do not need to do a literature reivew.
  \item If you are a PhD level student, you must work on this project by yourself and should write a brief literature review highlighting the closest papers to yours.
\end{itemize}



\end{document}
