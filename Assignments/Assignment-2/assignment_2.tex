\documentclass[12pt]{article}
\usepackage{../../lecture_notes}
\usepackage{../../math}
\usepackage{../../uark_colors}

\hypersetup{
  colorlinks = true,
  allcolors = ozark_mountains,
  breaklinks = true,
  bookmarksopen = true
}

\begin{document}
\begin{center}
  {\Huge\bf Assignment \#2}
  
  \smallskip
  {\large\it  ECON 5783 — University of Arkansas}

  \medskip
  {\large Prof. Kyle Butts}
\end{center}

These assignments should be completed in groups of 1 or 2 but submitted individually. My preference is for you to use Rmarkdown files to have your code, results, and your answers to the questions intermixed. Since I am not requiring you to code in R for these assignments, you can use latex or microsoft word to write up answers alternatively. Not that in these cases, I would like you to upload your code seperately. 

\section*{Theoretical Questions}

\begin{enumerate}
  \item Define the following conditional expectation function: $$
    p(n) \equiv \expec{\text{Home Value}_i}{\text{Num Rooms}_i = n},
  $$
  where $i$ denotes a given home in Massachusetts. 
  \begin{enumerate}
    \item In words, describe how to think about $p(4)$.
    
    \item Say you have a sample of parcels, how would you go about estimating $p(n)$? How would you estimate this in a regression?
    
    \item How does the ``fully-flexible'' conditional expecation function differ from a linear regression model where $\text{Home Value}_i = \text{Num Rooms}_i \beta + u_i$?
    
    \item Why might we not believe that $p(5) - p(4)$ be the causal effect of increasing from 4 to 5 rooms on homeprice?
  \end{enumerate}

  \item Now let's think about a more complicated conditional expectation of worker's wages as a function of sex (M/F) and having a college-degree (0/1):
  $$
    w(g, d) \equiv \expec{\text{Wages}_i}{\text{Gender}_i = g, \text{College Degree}_i = 1}
  $$
  \begin{enumerate}
    \item Say I include a set of indicator variables for gender and for whether or not the worker has a college degree in a regression. Describe a scenario where this regression model is not the conditional expecation function (hint: think about interactions).
  \end{enumerate}
\end{enumerate}



\section*{Coding Exercise}

This exercise involves a sample dataset of homes in Massachussets. A randomly selected sample of the dataset appears in this folder \texttt{data/MA\_parcels\_sample.parquet}. You can open this dataset using \texttt{arrow::read\_parquet("data/MA\_parcels\_sample.parquet")} after installing the arrow package in R. 

\begin{enumerate}
  \item Estimate the conditional expecation function from the theoretical question using regression (the \texttt{i} function from the \texttt{fixest} package will help you): $$
    p(n) \equiv \expec{\text{Home Value}_i}{\text{Num Rooms}_i = n}.
  $$
  What is the predicted change in home price from going from 4 to 5 rooms?

  \item Now estimate a linear regression of Home Value on the number of rooms. What is the predicted change in home price from going from 4 to 5 rooms? How does it differ from part (i)
  
  \item Now I want to use binscatter regression to estimate the relationship between the lot size of the home to the total value controlling (linearly) for the number of rooms. Use the \texttt{binsreg} package to do this. Use the setting \texttt{line = c(0, 0)} and \texttt{line = c(2, 2)}. 
  
  \item Do you think this is the causal effect of lot size on home value? Why or why not?
\end{enumerate}


\end{document}
