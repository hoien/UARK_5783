\documentclass[12pt]{article}
\usepackage{../../lecture_notes}
\usepackage{../../math}
\usepackage{../../uark_colors}

\hypersetup{
  colorlinks = true,
  allcolors = ozark_mountains,
  breaklinks = true,
  bookmarksopen = true
}

\begin{document}
\begin{center}
  {\Huge\bf Assignment \#3}
  
  \smallskip
  {\large\it  ECON 5783 — University of Arkansas}

  \medskip
  {\large Prof. Kyle Butts}
\end{center}

These assignments should be completed in groups of 1 or 2 but submitted individually. My preference is for you to use Rmarkdown files to have your code, results, and your answers to the questions intermixed. Since I am not requiring you to code in R for these assignments, you can use latex or microsoft word to write up answers alternatively. Not that in these cases, I would like you to upload your code seperately. 

\section*{Coding Exercise}

This assignment serves as an introduction to Monte-Carlo simulations and will assess your understanding of causal assumptions needed in different methods (today's is difference-in-means). Monte-Carlo simulations are a common strategy to assess the performance of estimators. The main idea is that you generate a dataset that either does or does not satisfy the underlying assumptions of a method and assess how the estimator does. 

We will generate the data that satisfies the conditional indepedence assumption, so that any of the estimators we have studied could possibly work. 

In the file \texttt{break\_the\_simulation\_2.qmd}, there is a complete example of a monte carlo simulation. Your task is to break the regression adjustment estimator *without* breaking the IPTW estimator. 


After doing so, describe what you did in the \texttt{\#\# Describe your thinking} section.


\end{document}
